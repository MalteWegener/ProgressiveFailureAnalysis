\documentclass{article}

\usepackage{natbib}

\title{Progressive Failure Analysis of a Compression Panel, by seperation of Components}
\author{Malte Wegener}
\begin{document}
	\maketitle
	\begin{abstract}
		This paper explains and demonstrates a method, to perform simple Progressive Failure Analysis on
		a stiffened metal Panel, to provide the basis for selecting design choices on this. It focuses on the method of Computation rather than the implications for Design.
	\end{abstract}

	\section{Introduction}
	Progressive Failure Analysis (PFA) is performed to gain insight into the modes of failure a specimen exhibits when exposed to a specific load. This can be done by Finite Element Analysis, which is very accurate, but on the other hand computationally expensive, this raises the need for faster methods of computation. This approach utilises a less accurate but faster approach to this problem.
	
	\section{Stress and Strain in the different Components}
	The specimen that is to be examined is clamped on both sides and compressed by a force, which acts on the centre of the Specimen, this causes various reactions in the Panel. The most obvious result of this compression is the compressive Stress in the Skin and the Stiffeners. A less obvious result is the bending Moment, which is induced by the Offset of the Force from the Neutral Axis of the stiffened panel.\\
	In order to calculate the respective stresses for the different components, the geometrical Properties of the Material have to be calculated, which are the Moment of Inertia, The Neutral Axis and the Area of each individual Component. These Calculations are trivial for the skin panel and will not be demonstrated in this Part. For the stiffeners a L-shape with equally sized legs is assumed. The Neutral Axis for this shape is given by, in which x is the length of the legs and t is the thickness of the material.\\
	\begin{equation} \label{Area of Stiffener}
		A = 2xt-t^2
	\end{equation}
	\begin{equation} \label{Neutral Axis of a Stiffener}
		N.A. = \frac{1}{4}*\frac{xt(x+t)-t^3}{xt-t^2}
	\end{equation}
	\begin{equation} \label{Moment of Inertia of a Stiffener}
		I_x = I_y = \frac{1}{12}*(xt(x^2+t^2)-t^4)+(\frac{x}{2}-N.A.)^2xt+(N.A.-\frac{t}{2})^2*xt-(N.A.-\frac{t}{2})^2*t^2
	\end{equation}
	
	When calculating the behaviour of a section of the skin between two stiffeners, Hooke's Law predicts an expansion perpendicular to the applied Force, this expansion However is prohibited by the Stiffeners. By solving Hooke's Law under the premise that $\sigma_y = 0$ it can be shown, that
	
	\begin{equation} \label{Comp1}
		\sigma_y = v*\sigma_x
	\end{equation}
	this yields for the strain of the skin
	
	\begin{equation} \label{strain Skin}
		\epsilon_x = \frac{\sigma_x}{E}*(1-v^2)
	\end{equation}
	
	Furthermore, due to the usage of different materials in different Components, it is conjectured that the deformation of each Component in the longitudinal Direction is equal, up to the point of buckling. 
	
	\begin{equation} \label{DefComp}
		\epsilon_{skin} = \epsilon_{stiffener}
	\end{equation}
	
	
	From these Calculations, the different locations of Buckling can be deduced.
	\begin{enumerate}
		\item Lateral buckling of the skin
		\item Longitudinal buckling of the Panel
		\item Buckling due to the bending Moment
		\item Interrivet buckling of the skin
	\end{enumerate}

	\section{Solver}
	For efficient and accurate solving of these compatibility equations and deformation equations, a two stage solver is used. In the first iteration the stress is distributed in a way, which strains obey equation \ref{strain Skin} and \ref{DefComp}.\\
	In parallel the critical Force for buckling is calculated for each of the previously mentioned Locations for buckling.
	By comparing these results it is decided, if a component will buckle. As soon as a component begins to buckle, the stress in this component stays constant, which is taken into Account during the next Calculation.
	
\end{document}
















